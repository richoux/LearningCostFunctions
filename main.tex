\documentclass{article}
\pdfpagewidth=8.5in
\pdfpageheight=11in
% The file ijcai20.sty is NOT the same than previous years'
\usepackage{ijcai20}

\usepackage[english]{babel}
\usepackage[T1]{fontenc}

% Use the postscript times font!
\usepackage{times}
\usepackage{soul}
\usepackage{url}
\usepackage[hidelinks]{hyperref}
\usepackage[utf8]{inputenc}
\usepackage[small]{caption}
\usepackage{graphicx}
\usepackage{amsmath}
\usepackage{amsthm}
\usepackage{booktabs}
\usepackage{algorithm}
\usepackage{algorithmic}
\urlstyle{same}

\usepackage{xspace}
\usepackage[dvipsnames, table]{xcolor}
\usepackage{color}
\usepackage{amsfonts}
\usepackage{adjustbox}
%\usepackage[ruled,linesnumbered]{algorithm2e}
\usepackage{hyperref}
\hypersetup{
  colorlinks   = true, %Colours links instead of ugly boxes
  urlcolor     = blue, %Colour for external hyperlinks
  linkcolor    = blue, %Colour of internal links
  citecolor   = blue %Colour of citations
}

%\title{Automatic Cost Function Generation ('learning' in the title?)}
\title{Automatic Learning of Cost Functions to Help Modeling Cost Function Networks}

\iffalse
\author{
Florian Richoux$^{1,2}$
\and
Jean-François Baffier$^3$
\affiliations
$^1$JFLI, CNRS, NII\\
$^2$Université de Nantes\\
$^3$RIKEN AIP
\emails
florian.richoux@polytechnique.edu,
jf@baffier.fr
}
\fi

\newcommand{\ie}{\textit{i.e.}}
\newcommand{\cp}{\textsc{CP}\xspace}
\newcommand{\csp}{\textsc{CSP}\xspace}
\newcommand{\cop}{\textsc{COP}\xspace}
\newcommand{\cfn}{\textsc{CFN}\xspace}
\newcommand{\cf}{\textsc{cf}\xspace}
\newcommand{\wcsp}{\textsc{WCSP}\xspace}
\newcommand{\ghost}{\textsc{GHOST}\xspace}

\newcommand{\flo}{\textcolor{blue}{\bf Flo}\xspace}
\newcommand{\jf}{\textcolor{red}{\bf JF}\xspace}

\begin{document}

\maketitle

\begin{abstract}
  Cost  Function  Networks  (\cfn)   are  a  formalism  in  Constraint
  Programming  to  model  combinatorial satisfaction  or  optimization
  problems.   By associating  a function  to each  constraint type  to
  evaluate the quality  of an assignment, it  extends the expressivity
  of regular \csp/\cop formalisms but at  a price of making harder the
  problem    modeling.     Indeed,     in    addition    to    regular
  variables/domains/constraints sets,  one must provide a  set of cost
  functions (\cf) that are not always easy to define.  Here we propose
  a  method to  automatically learn  a \cf  of a  constraint, given  a
  function deciding if  assignments are valid or not.  This  is to the
  best  of our  knowledge  the first  attempt  to automatically  learn
  $\cf$s.  Our  method aims to  learn $\cf$s in a  supervised fashion,
  trying to  reproduce the Hamming  distance, by using a  variation of
  neural  networks  allowing us  to  get  explainable results,  unlike
  regular artificial neural networks.  We experiment it on 5 different
  constraints  to   show  its  versatility.   Experiments   show  that
  functions  learned on  small  dimensions scale  on high  dimensions,
  outputting a  perfect or near-perfect Hamming  distance.  Our method
  can be  used to  automatically generate $\cf$s  and then  having the
  expressivity  of  \cfn  with  the  same  modeling  effort  than  for
  \csp/\cop.
\end{abstract}

\section{Introduction}\label{sec:introduction}

TODO: CP  est difficile à  modéliser~\cite{Puget2004,Wallace2003}, CFN
ajoute une structure  utile au solveur mais est  également difficile à
modéliser  (même plus)  : générer  automatiquement les  CF pour  aider
l'utilisateur à modéliser son problème.  Pour palier à la modélisation
CP, on se repose sur  l'utilisation de méta-heuristiques qui n'ont pas
besoin       d'une        sélection       fine        des       bonnes
contraintes. \cite{AMJFH2011,Bessiere2015,CBLS}

Motivation : pas forcément facile ni intuitif de trouver une bonne CF.


\subsection{Of the interest of \cfn over classic CSP}
As shown in  [previous studies], using a cost function  to guide a CSP
solver, even for a satisfaction problem, generally greatly improve the
convergence speed to a solution.  It is important to note that using a
\cfn with  a manual input of  the cost functions (that  is the current
general usage of such solvers) greatly increase the difficulty for the
user.

The aim of  this research is to  provide a set of tools  to compute or
estimate those  cost functions  in a  way that  provides the  user the
following guarantees: CSP $<$ CFN-auto $\leq$ CFN-manual.

We hope  to eventually  provides methods that  can lead  to CFN-manual
$\leq$ CFN-auto for some (if not not most) families of constraints.

\section{Preliminaries}\label{sec:preliminaries}
Topo sur CPPN \jf \cite{CPPN}.

In the  literature, Cost  Function Networks  (\cfn) and  Weighted \csp
(\wcsp)  are  synonyms~\cite{Zytnicki2009,Bessiere2011}.  Some  papers
like~\cite{Allouche2012} present  \cfn to  be the formalism  and \wcsp
the  problem of  finding an  assignment minimizing  the combined  cost
function of a given \cfn instance. Since  it is rarely a good think in
Science to have two different names for the same notion, we start this
paper by proposing clear, distinct definitions of \cfn and \wcsp.

\subsection{Definitions of \wcsp and \cfn} 

We propose  to keep the definition  of a \wcsp from  the literature: a
\textbf{\wcsp} is a tuple ($V$,$D$,$C$,$F$)  where $V$ is a finite set
of variables,  $D$ a finite set  of domains, one for  each variable in
$V$, each domain  being the set of  values a variable can  take, $C$ a
finite set  of constraints  over variables  in $V$,  determining which
combinations of values in $D$ are allowed or forbidden to describe our
problem, and finally a finite set  $F$ composed of cost functions, one
for  each constraint  in $C$.   Let's  denote by  $D_c$ the  Cartesian
product of  the domain of  variables involved  in a constraint  $c \in
C$. The cost function $f_c \in  F$ associated to the constraint $c$ is
a     function     $f_c:     D_c    \rightarrow     \{0,k\}$     where
$k \in \mathbb{N} \cup \{\infty\}$ is the special cost for assignments
violating the  constraint $c$. Thus, an  assignment $\alpha$ satisfies
the constraint  $c$ iff  $f_c(\alpha) < k$  holds. The  function $f_c$
allows  us to  rank assignments  and  to express  softness within  the
constraint~$c$.

A \textbf{\cfn} is also defined  by a tuple ($V$,$D$,$C$,$F$) with the
same  sets  as   \wcsp.   The  difference  we  propose   lies  in  the
interpretation of cost functions $f_c$.  In this paper, cost functions
defined   in   a   \cfn    are   functions   $f_c:   D_c   \rightarrow
\mathbb{R}^+$. An assignment $\alpha$ satisfies the constraint $c$ iff
$f_c(\alpha) = 0$ holds. All  other strictly positive outputs of $f_c$
lead to forbidden assignments. Therefore, \cfn is considering hard (or
crisp) constraints  only, unlike  \wcsp. Strictly positive  outputs of
$f_c$ are then interpreted  like preferences over invalid assignments:
the closer $f_c(\alpha)$ is to 0, the closer $\alpha$ is to be a valid
assignment.

In the same  way a \csp instance  is a network of  constraints, \ie, a
networks of  predicates expressing if  an assignment satisfies  or not
each constraint, a \cfn instance  is a network of functions expressing
if  an assignment  satisfies the  constraints or  how close  it is  to
satisfy them. Thus,  this formalism \cfn allows us to  express a finer
structure about the  problem, since we furnish with  cost functions an
ordered  structure  over  invalid  assignment  a  solver  can  exploit
efficiently   to   improve  the   search.   We   illustrate  this   in
Section~\ref{sec:xp}.

Observe we  can also deal  with optimization  problems with a  \cfn by
adding  to  the  tuple  ($V$,$D$,$C$,$F$)  an  objective  function  to
optimize.

% \wcsp and \cfn are defined by close but different flavors throughout
% the litterature. To make the most  of both, we chose to consider the
% following definitions for those two concepts.

%  \paragraph{\wcsp}   is  considering   soft  constraints,   where  a
% constraint is unsatisfyed iif its associated cost function outputs a
% value below a given threshold $k$ (can be infinite).

%  \paragraph{\cfn}  is  considering   hard  constraints  only.   Like
% constraint  networks helping solvers  to find solutions by  giving a
% structure of  the problem, CFN gives, in addition  of the constraint
%  network,  a structure  on  configurations  to  help the  solver  to
% determine if an unsatisfying configuration  is near to be a solution
% or not.

From those  definition, we  can consider \wcsp  and the  \cfn variants
(satisfaction  and  optimization)  to have  different  expressiveness:
\cfn-sat $<$ \wcsp $<$ \cfn-opt

In  this study,  we deliberately  chose  to focus  on \cfn  as we  can
naturally transform any \wcsp into \cfn-opt.

\section{Method design \jf}\label{sec:method}
\flo{Préciser ce qu'est notre la loss function}

From  head to  toes,  how our  framework  is designed  (interpolation,
pre-processing and such) to fit a solver (here GHOST).

Modèles (qui n'ont pas marché) :
\begin{itemize}
\item CF comme combinaison linéaire de sinus
\item CF en CPPN de fonctions simples (sinus, tanh, sigmoid, ...)
\item CF apprise par CFN (lol) avec smoothness comme fonction objectif.
\item Relaxation du SL en regardant seulement l'ordre de Hamming avec le modèle CPPN.
\end{itemize}

\flo: insister  sur le  fait que, malgrè  le fait  que l'apprentissage
supervisé   se   fasse   à   proprement   parler   sur   des   données
``configurations + coût de Hamming''  (réel ou estimé), un utilisateur
ne  fournit QUE  le concept  de la  contrainte visée,  et non  pas ces
données en  question.  Nous fournissons  les outils pour  produire les
données ``configurations + coût de Hamming'' à partir d'un concept, et
cette   production   peut   être   automatisée   dans   la   procédure
d'apprentissage, et est donc transparente à l'utilisateur.

\subsection{Scaling Issues}\label{subsec:issues}
Scaling  our  results  on  small instances  is  challenging.  In  this
subsection we cover 4 possible approaches that are either unsuccessful
or not yet efficient.

\paragraph{Learning a cost function on small space}
For general constraints, there is no generic idea to extend a known cost
function from a small space to a higher dimension one.

\section{Experiments}\label{sec:xp}

To  show  the versatility  of  our  method, we  tested  it  on 5  very
different constraints:  AllDifferent, Ordered,  LinearSum, NoOverlap1D
and Minimum.   According to XCSP specifications~\cite{xcsp}  (see also
\href{http://xcsp.org/specifications}{xcsp.org/specifications}), those
global  constraints   below  to   4  different   families:  Comparison
(AllDifferent    and     Ordered),    Counting/Summing    (LinearSum),
Packing/Scheduling  (NoOverlap1D)  and Connection  (Minimum).   Always
according to XCSP specifications, these 5 constraints are among the 20
most popular and  common constraints.  We give a  brief description of
those 5 constraints below:

\paragraph{AllDifferent} ensures  that variables must all  be assigned
to different values.
\paragraph{Ordered} ensures  that an  assignment of variables  must be
ordered, given a total order. In this paper, we choose the total order
$\leq$. Thus, for all indexes $i,j \in \{1,n\}$ with $n$ the number of
variables, $i < j$ implies $x_i \leq x_j$.
\paragraph{LinearSum}       ensures       that      the       equation
$x_1 + x_2 +  \ldots + x_n = p$ holds, with the  parameter $p$ a given
integer.
\paragraph{NoOverlap1D}  is considering  variables as  tasks, starting
from a  certain time (their  value) and each  with a given  length $p$
(their  parameter).    The  constraint  ensures  that   no  tasks  are
overlapping,  \ie, for  all indexes  $i,j  \in \{1,n\}$  with $n$  the
number   of   variables,  we   have   $x_i   +   p_i  \leq   x_j$   or
$x_j + p_j \leq  x_i$.  To have a simpler code,  we have considered in
this paper that all tasks have the same length $p$.
\paragraph{Minimum} ensures  that the  minimum value of  an assignment
verifies  a given  numerical condition.  In this  paper, we  choose to
consider that  the minimum value must  be greater than or  equals to a
given parameter $p$.

\subsection{Experimental protocols}

To have  experimental evidences  of the efficiency  of our  method, we
conducted several different experiments.

All experiments have  been conducted on a computer with  a CPU Core i7
6700K and  48GB of RAM,  running on  Ubuntu 18.04. Programs  have been
compiled with GCC with the 03 optimization option.

\subsubsection{Experiment 1: scaling}

The first  one consists in  learning cost  functions on a  very small,
complete  constraint  space  (\flo:  on doit  définir  ce  qu'est  cet
espace), composed  of about 500  configurations (\flo: être  clair sur
les définitions de configuration et solution).  It is then possible to
compute  the  Hamming distance  between  each  configuration with  its
closest  solution,  \ie, the  solution  that  requires to  modify  the
smallest number  of variables  from the considered  configuration. The
goal of this  experiment is to show that learned  cost functions scale
to  high-dimensional constraints,  making  sufficient the  use of  our
method on small  constraint instances to get  efficient cost functions
on any number of variables.

We run  100 cost  function learnings on  the same  complete constraint
space, for  each of  the 5 constraints  presented above  (\flo: écrire
dans  la section  3 que  la loss  est la  différence entre  le Hamming
estimé et le Hamming réel). We then analyze
the frequency of  cost functions we get and  compute the discrepancies
of the  most frequent ones to  the Hamming distance, over  100 sampled
configurations  composed of  100  variables on  domains  of size  100,
belonging  to  constraint  spaces   of  size  $10^{200}$  (compare  to
constraint spaces of size around 500 used to learn cost functions).

\flo: Apprendre une CF sur une petit instance, et la comparer avec 1. les
métriques  sur cette  même instance,  2.  les métriques  sur une  plus
grande instance et 3. Une CF apprise sur une plus grande instance.

\subsubsection{Experiment 2: learning over incomplete spaces}

If, for any reasons, it is not possible to build a complete constraint
space, a robust system must be  able to learn effective cost functions
on  large,  incomplete spaces  where  the  exact Hamming  cost  (\flo:
Hamming cost of config $c$ =  Hamming distance of $c$ with its closest
solution) of their configurations is unknown.

In this experiment,  we sample $k$ solutions and  $k$ non-solutions on
large  constraint  spaces,  approximate   the  Hamming  cost  of  each
non-solution  by computing  their  Hamming distance  with the  closest
solution among  the $k$ ones, and  learn cost functions on  these $2k$
configurations and their estimated Hamming  cost. Then, we compare the
frequency of cost  functions we obtain this way with  the frequency of
cost functions  learned on  small and  complete constraint  spaces, as
well as the discrepancies of the most frequent cost functions obtained
both ways to the real Hamming  distance. \flo: pas super clair ; faire
un  effort pour  mieux  présenter  la métrique  de  réussite de  cette
expérimentation. 

\flo:  Comparaison  des  CF  obtenues avec  différent  pourcentage  de
l'espace des configurations samplée.

\flo: Généralisation  : apprendre  une CF  pour une  instance avec  certains
paramètres  et  tester sur  d'autres  paramètres  (et/ou sur  d'autres
tailles aussi)

\subsubsection{Experiment 3: using learned $\cf$s to solve problems}

Here, we use learned cost  functions for AllDifferent and LinearSum in
\cfn modeling  3 problems:  Sudoku (using AllDifferent),  Magic Square
(using  LinearSum)  and Killer  Sudoku  (using  both AllDifferent  and
LinearSum). We use  a local search solver to solve  these problems and
consider the mean run-time as a metric to compare pure \csp models (so
without cost functions),  \cfn models with learned  cost functions and
\cfn models with handmade cost functions.

\paragraph{Sudoku} is a puzzle game presented like a $9 \times 9$ grid
where each cell of the grid must be  filled up with a number from 1 to
9,  such that  each row  and each  column contains  exactly once  each
number. In addition, the grid is composed of 9 smaller squares of size
$3  \times 3$  which  must also  be filled  with  each number  exactly
once. In other words, all numbers  of each row, column and square must
be  different,   which  is  perfectly  modeled   by  the  AllDifferent
constraint.   Usually, the  grid  is pre-filled  with  a few  numbers,
preventing from  finding a trivial  solution.  In this paper,  to have
randomly  generated Sudoku  instances, we  have pre-filled  the entire
grid randomly with the expected total number of 1s, 2s, and so on, and
ask  the solver  to  find a  permutation  satisfying all  AllDifferent
constraints described above.

\paragraph{Magic Square} is a $n \times n$ grid that must be filled up
with all number from 1 to $n^2$ (thus, all numbers must appear exactly
once in the grid), such that the  sum of each row, each column and the
two diagonals must be equal to a constant $c$.  We can avoid using the
AllDifferent  constraint by  randomly  filling up  the  grid with  all
expected numbers and ask the solver to find a correct permutation. The
constraints over  the rows, columns  and the two diagonal  are modeled
with LinearSum,  since the  value of  $c$ only depends  on $n$  and is
known to be $c = n(n^2 + 1)/2$.

\paragraph{Killer Sudoku}  is the same than  Sudoku but in such  a way
that the  grid is  paved with  blocks of  cells, named  cages, usually
composed of 2, 3  or 4 cells.  To each cage  is associated an integer,
and  the sum  of  numbers in  theirs  cells must  be  equals to  their
integer. A  killer Sudoku  instance starts with  an empty  grid, cages
preventing from  trivial solutions. AllDifferent constraints  are used
to model  the regular Sudoku rules  of this puzzle game  and LinearSum
constraints are modeling cages.

\flo: Comparaison des  CF obtenues sur des  benchs avec des CF  handmades et
sans CF

\subsection{Results}
Figure et tableaux \jf.

In this part, we  will denote by $n$ the number  of variables, $d$ the
domain size  and $p$  the value of  an eventual  parameter. Constraint
instances are denoted by \textit{name-n-d[-p]}.

\subsubsection{Experiment 1}

% Since many different cost functions can be learn over 100 runs for the
% same constraint instance, we will only consider cost functions learned
% at least 10\% of the time.

Like  written in  Section~\ref{sec:method}, our  loss function  is the
absolute value of the difference  between the expected Hamming cost of
a  configuration $c$  and  the  Hamming cost  estimated  for the  cost
function on $c$. The loss function is then normalized with the size of
the constraint space  used for training, giving us  the training error
of the constraint space, \ie,  the average difference between expected
and  estimated  Hamming costs.   Thus,  a  cost  function $f$  with  a
training error  of 2  means that  $f$ estimations  on configurations
used for training are in average +2 or -2 from the real Hamming cost.

% In  this experiment,  we  learn 100  times a  cost  function for  each
% constraint instance, and  we compute the mean of  their training error
% over  the  whole  configuration  space.  The  training  error  is  the
% difference between the expected Hamming cost and the estimated Hamming
% cost over a  configuration from the configuration space  used to train
% the  function.  For instance,  if  a  cost  function  outputs 5  on  a
% configuration $c$  when the expected  Hamming cost  of $c$ is  3, this
% function has an error of 2. 

In  this experiment,  we  learn 100  times a  cost  function for  each
constraint instance. Table\ref{tab:cf_small} shows for each constraint
instance the median  and mean training errors of the  100 learned cost
functions,  as well  as  the  training error  of  the most  frequently
learned  cost function,  and  its frequency  in  parenthesis. In  this
experience,   the   most   frequently  learned   cost   function   was
systematically the one with the lowest training error.

Notice that the most frequent cost function for all\_diff-4-5, learned
97 over 100 runs, is actually two equivalent functions $f_1$ and $f_2$
expressed differently,  respectively learned 58  and 39 times.   For a
configuration  $c =  (c_1, c_2,  \ldots,  c_n)$, $f_1$  and $f_2$  are
defined as follows:
\begin{align*}
  f_1(c) &:= Count_{\geq 0}( Number\ of\ c_j\ s.t.\ i<j\ and\ c_i=c_j)\\
  and&\\
  f_2(c) &:= Count_{\geq 0}( Number\ of\ c_j\ s.t.\ i>j\ and\ c_i=c_j)
\end{align*}

Due  to space  limitation, other  learned cost  functions will  not be
detailed in this paper.   We invite the reader to look  at them in the
code                           repository                           at
\href{https://anonymous.4open.science/r/50ffb3e8-8918-4f59-9b8d-ef80060585ef/}{anonymous.4open.science/r/50ffb3e8-8918-4f59-9b8d-ef80060585ef}
by following the procedure described in the README file.

% \paragraph{all\_diff-4-5} Two cost functions were 

% \paragraph{ordered-4-5} 

% \paragraph{linear\_sum-3-8-12}

% \paragraph{no\_overlap-3-8-2}

% \paragraph{minimum-4-5-3}

\begin{table}
  \centering
\begin{tabular}{|l|l|l|l|}
  \hline
  Constraints & median & mean & most freq.\\
  \hline
  all\_diff-4-5 & 0 & 0.03 & 0~~~~~~(97)\\
  ordered-4-5 & 0.08 & 0.08 & 0.08~(100)\\
  linear\_sum-3-8-12 & 0.01 & 0.05 & 0.01~(74)\\
  no\_overlap-3-8-2 & 0.14 & 0.19 & 0.11~(50)\\
  minimum-4-5-3 & 0 & 0.04 & 0~~~~~~(88)\\
  \hline
\end{tabular}
\caption{Median, mean and  most frequent training error  over 100 runs
  (with frequency in parenthesis) of learned cost functions over small
  complete constraint spaces.}
\label{tab:cf_small}
\end{table}

Learning cost functions over small complete constraint spaces of about
500 configurations takes about 10 seconds on our hardware.

Table~\ref{tab:cf_small} shows very good performances, but it might be
due  to overfitting  on those  small constraint  spaces.  To  check if
learned  cost functions  do not  overfit and  can scale  to constraint
instances on higher dimensions, we use the most frequent cost function
learned on each  constraint for estimating the Hamming  cost of 20,000
random configurations sampled from high-dimensional constraint spaces.

For AllDifferent, LinearSum and Minimum, it  is easy to define by hand
a function computing the Hamming cost of any configuration $c$ without
generating  the  whole constraint  space.  For  these constraints,  we
tested the  corresponding cost function  on spaces with  100 variables
and domains of size 100.

For  Ordered   and  NoOverlap1D,  since  these   two  constraints  are
intrinsically  combinatoric, finding  a function  computing the  exact
Hamming  cost of  any  configuration is  not  trivial.  Therefore,  we
sampled 10,000 solutions and 10,000 non-solutions in constraint spaces
of   ordered-12-18   (so    $18^{12}$   configurations,   \ie,   about
$1.15\times         10^{15}$)          and         no\_overlap-10-35-3
($35^{10}  \simeq   2.75\times  10^{15}$  configurations).    We  then
approximate  the Hamming  cost of  each non-solution,  considering the
closest solution among the 10,000 sampled solutions.

\begin{table}
  \centering
\begin{tabular}{|r|r|r|r|r|}
  \hline
  all-diff & ord & lin-sum & no-ol & min\\
  \hline
  0 & 1.274 & 0.001 & 2.686 & 0\\
  \hline
\end{tabular}
\caption{Mean error  over 20,000 configurations in  high dimensions of
  learned cost functions over small complete constraint spaces.}
\label{tab:cf_scale}
\end{table}

Table\ref{tab:cf_scale} ...

\subsubsection{Experiment 2}

\subsubsection{Experiment 3}
\flo:  dire que  le but  ici  n'est pas  de  casser du  bench mais  de
comparer, même avec un solver  pas optimisé, les temps de résolutions
sans \cf, avec \cf apprise et avec \cf maison.

\section{Discussions}

Résultats améliorables  (notamment en  terme de  stabilité d'obtention
d'une CF optimale) en optimisant le GA, ce que nous n'avons pas fait.

Avantage  de  notre  modèle  :  les  CF  trouvées  sont  intelligibles
(contrairement aux NN). On peut même les coder en dur si l'on préfère,
plutôt que de calculer la CF avec un run forward du CPPN.

Peut être utilisé pour de l'aide à la décision, cad aider à définir un
CF à la main.

Autre avantage  : les CF  apprises sont robustes car  indépendantes de
l'instance des contraintes (taille et valeurs des paramètres)

La  simplification CPPN  où  les  poids sont  booléens  est aussi  une
avantage  pour  simplifier  la représentation  (et  compréhension,  et
reproduction) de la CF par rapport à des poids dans [0,1].

CPPN facile  à modifier (ses  opérations, notamment) pour  des besoins
spécifiques, des problèmes ou des contraintes particulières.


\section{Conclusion and perspectives}\label{sec:conclusion}

Faire du  RL plutôt  que du  SL. Pas  sûr que  Hamming soit  une bonne
métrique. Cela permettra de trouver des CF mieux adaptées aux algos.


\bibliographystyle{named}
\bibliography{main}

\end{document}
