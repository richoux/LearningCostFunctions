\def\year{2020}\relax
%File: formatting-instruction.tex
\documentclass[letterpaper]{article} % DO NOT CHANGE THIS
\usepackage{aaai20}  % DO NOT CHANGE THIS
\usepackage{times}  % DO NOT CHANGE THIS
\usepackage{helvet} % DO NOT CHANGE THIS
\usepackage{courier}  % DO NOT CHANGE THIS
\usepackage[hyphens]{url}  % DO NOT CHANGE THIS
\usepackage{graphicx} % DO NOT CHANGE THIS
\urlstyle{rm} % DO NOT CHANGE THIS
\def\UrlFont{\rm}  % DO NOT CHANGE THIS
\usepackage{graphicx}  % DO NOT CHANGE THIS
\frenchspacing  % DO NOT CHANGE THIS
\setlength{\pdfpagewidth}{8.5in}  % DO NOT CHANGE THIS
\setlength{\pdfpageheight}{11in}  % DO NOT CHANGE THIS

\usepackage{xspace}
\usepackage[dvipsnames, table]{xcolor}
\usepackage{amsmath}
\usepackage{adjustbox}
\usepackage[ruled,linesnumbered]{algorithm2e}

\usepackage[english]{babel}
\usepackage[T1]{fontenc}
\usepackage[utf8]{inputenc}

%PDF Info Is REQUIRED.
% For /Author, add all authors within the parentheses, separated by commas. No accents or commands.
% For /Title, add Title in Mixed Case. No accents or commands. Retain the parentheses.
 \pdfinfo{
/Title (Automatic Cost Function Generation)
/Author (Florian Richoux, Jean-Fran\c{c}ois Baffier)
} %Leave this
% /Title ()
% Put your actual complete title (no codes, scripts, shortcuts, or LaTeX commands) within the parentheses in mixed case
% Leave the space between \Title and the beginning parenthesis alone
% /Author ()
% Put your actual complete list of authors (no codes, scripts, shortcuts, or LaTeX commands) within the parentheses in mixed case.
% Each author should be only by a comma. If the name contains accents, remove them. If there are any LaTeX commands,
% remove them.

% DISALLOWED PACKAGES
% \usepackage{authblk} -- This package is specifically forbidden
% \usepackage{balance} -- This package is specifically forbidden
% \usepackage{caption} -- This package is specifically forbidden
% \usepackage{color (if used in text)
% \usepackage{CJK} -- This package is specifically forbidden
% \usepackage{float} -- This package is specifically forbidden
% \usepackage{flushend} -- This package is specifically forbidden
% \usepackage{fontenc} -- This package is specifically forbidden
% \usepackage{fullpage} -- This package is specifically forbidden
% \usepackage{geometry} -- This package is specifically forbidden
% \usepackage{grffile} -- This package is specifically forbidden
% \usepackage{hyperref} -- This package is specifically forbidden
% \usepackage{navigator} -- This package is specifically forbidden
% (or any other package that embeds links such as navigator or hyperref)
% \indentfirst} -- This package is specifically forbidden
% \layout} -- This package is specifically forbidden
% \multicol} -- This package is specifically forbidden
% \nameref} -- This package is specifically forbidden
% \natbib} -- This package is specifically forbidden -- use the following workaround:
% \usepackage{savetrees} -- This package is specifically forbidden
% \usepackage{setspace} -- This package is specifically forbidden
% \usepackage{stfloats} -- This package is specifically forbidden
% \usepackage{tabu} -- This package is specifically forbidden
% \usepackage{titlesec} -- This package is specifically forbidden
% \usepackage{tocbibind} -- This package is specifically forbidden
% \usepackage{ulem} -- This package is specifically forbidden
% \usepackage{wrapfig} -- This package is specifically forbidden
% DISALLOWED COMMANDS
% \nocopyright -- Your paper will not be published if you use this command
% \addtolength -- This command may not be used
% \balance -- This command may not be used
% \baselinestretch -- Your paper will not be published if you use this command
% \clearpage -- No page breaks of any kind may be used for the final version of your paper
% \columnsep -- This command may not be used
% \newpage -- No page breaks of any kind may be used for the final version of your paper
% \pagebreak -- No page breaks of any kind may be used for the final version of your paperr
% \pagestyle -- This command may not be used
% \tiny -- This is not an acceptable font size.
% \vspace{- -- No negative value may be used in proximity of a caption, figure, table, section, subsection, subsubsection, or reference
% \vskip{- -- No negative value may be used to alter spacing above or below a caption, figure, table, section, subsection, subsubsection, or reference

\setcounter{secnumdepth}{1} %May be changed to 1 or 2 if section numbers are desired.

% The file aaai19.sty is the style file for AAAI Press
% proceedings, working notes, and technical reports.
%
\setlength\titlebox{2.5in} % If your paper contains an overfull \vbox too high warning at the beginning of the document, use this
% command to correct it. You may not alter the value below 2.5 in
\title{Automatic Cost Function Generation}
%Your title must be in mixed case, not sentence case.
% That means all verbs (including short verbs like be, is, using,and go),
% nouns, adverbs, adjectives should be capitalized, including both words in hyphenated terms, while
% articles, conjunctions, and prepositions are lower case unless they
% directly follow a colon or long dash
\author{Florian Richoux\textsuperscript{\rm 1,2}\\
  \textsuperscript{\rm 1}JFLI, CNRS, NII\\
  \textsuperscript{\rm 2}LS2N, Université de Nantes\\
  % If you have multiple authors and multiple affiliations
% use superscripts in text and roman font to identify them. For example, Sunil Issar,\textsuperscript{\rm 2} J. Scott Penberthy\textsuperscript{\rm 3} George Ferguson,\textsuperscript{\rm 4} Hans Guesgen\textsuperscript{\rm 5}. Note that the comma should be placed BEFORE the superscript for optimum readability
florian.richoux@univ-nantes.fr % email address must be in roman text type, not monospace or sans serif
\And Jean-Fran\c{c}ois Baffier\textsuperscript{\rm 3}\thanks{Financement pour Jeff}\\
\textsuperscript{\rm 3}RIKEN AIP\\
%2275 East Bayshore Road, Suite 160\\
%Palo Alto, California 94303\\
jf@baffier.fr % email address must be in roman text type, not monospace or sans serif
}

\newcommand{\ie}{\textit{i.e.}}
\newcommand{\cp}{\textsc{CP}\xspace}
\newcommand{\csp}{\textsc{CSP}\xspace}
\newcommand{\cop}{\textsc{COP}\xspace}
\newcommand{\cfn}{\textsc{CFN}\xspace}
\newcommand{\wcsp}{\textsc{WCSP}\xspace}
\newcommand{\ghost}{\textsc{GHOST}\xspace}

\begin{document}

\maketitle

\begin{abstract}
A formidable abstract that shows off how much our contribution is efficient and user-friendly. Oh, and we have impressive experiments too. Be sure to subcribe our GitHub repository.
\end{abstract}

\section{Introduction}\label{sec:introduction}

TODO: CP est difficile à modéliser~\cite{Puget2004,Wallace2003}, CFN ajoute une structure utile au
solveur  mais est  également  difficile à  modéliser  (même plus)  :
générer  automatiquement   les  CF  pour  aider   l'utilisateur  à
modéliser  son problème.  Pour palier  à la  modélisation CP,  on se
repose sur  l'utilisation de méta-heuristiques qui  n'ont pas besoin
d'une sélection fine des bonnes contraintes. \cite{AMJFH2011,Bessiere2015,CBLS}

\subsection{Of the interest of \cfn over classic CSP}
As shown in [previous studies], using a cost function to guide a CSP solver, even for a satisfaction problem, generally greatly improve the convergence speed to a solution.
It is important to note that using a \cfn with a manual input of the cost functions (that is the current general usage of such solvers) greatly increase the difficulty for the user.

The aim of this research is to provide a set of tools to compute or estimate those cost functions in a way that provides the user the following guarantees:
CSP $<$ CFN-auto $\leq$ CFN-manual.

We hope to eventually provides methods that can lead to CFN-manual $\leq$ CFN-auto for some (if not not most) families of constraints.



\section{Preliminaries}\label{sec:preliminaries}
TODO: Formalisation    WCSP     vs    CFN     (section    5     du
draft). \cite{Bessiere2011,LK2014}
\subsection{Differentiating \wcsp and \cfn}
\wcsp and \cfn are defined by close but different flavors throughout the litterature. To make the most of both, we chose to consider the following definitions for those two concepts.

\paragraph{\wcsp} is  considering soft  constraints, where  a constraint  is
  unsatisfyed iif its associated cost function outputs a value below a
  given threshold $k$ (can be infinite).

\paragraph{\cfn} is  considering  hard constraints  only.  Like  constraint
  networks helping solvers to find  solutions by giving a structure of
  the problem,  CFN gives,  in addition of  the constraint  network, a
  structure on  configurations to help  the solver to determine  if an
  unsatisfying configuration is near to be a solution or not.

From those definition, we can consider \wcsp and the \cfn variants (satisfaction and optimization) to have different expressiveness: \cfn-sat $<$ \wcsp $<$ \cfn-opt

In this study, we deliberately chose to focus on \cfn as we can naturally transform any \wcsp into \cfn-opt.

\section{Theoretical Design}\label{sec:theory}
From head to toes, how our framework is designed (interpolation, pre-processing and such) to fit a solver (here GHOST).

\subsection{Scaling Issues}\label{subsec:issues}
Scaling our results on small instances is challenging. In this subsection we cover 4 possible approaches that are either unsuccessful or not yet efficient.

\paragraph{Learning a cost function on small space}
For general constraints, there is no generic idea to extend a known cost
function from a small space to a higher dimension one.

\section{Experimental Results}\label{sec:xp}
Presenting the aim of those experiments, the choice of the datasets, and the results itself.

\subsection{Automatic Generation of Cost Functions}
\label{subsec:xpgeneration}

\subsection{Comparison with crafted cost functions}\label{subsec:xpcomparison}

\section{Conclusion}\label{sec:conclusion}


% \nocite{*}
%
\bibliographystyle{aaai}
\bibliography{main}

\end{document}
